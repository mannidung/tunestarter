\documentclass[10pt]{book}
\usepackage{PREAMBLE}
%\usepackage[showframe,paper=a4paper,margin=1in]{geometry}
\usepackage[paper=a4paper,margin=0.5in]{geometry}
\usepackage[generate,ps2eps]{abc}
\usepackage{mathptmx}

%% Formatting
\renewcommand*{\familydefault}{\sfdefault} % Set default font to sans serif
\renewcommand\thesection{} % Remove numbering of section
\renewcommand\thesubsection{} % Remove numbering of subsections
\renewcommand*\contentsname{Contents} % Change "contents" to "innehåll"
\renewcommand*\chaptername{Chapter}
\usepackage[export]{adjustbox}
\usepackage{cmbright} % Sets font
\usepackage{enumitem} % Together with next line it sets the enumerations and lists to no separation
\setlist{nosep}
\usepackage{wrapfig} % For right aligned image with text to the left
% ref packages
\usepackage{nameref}
% folowing  must be in this order
\usepackage{varioref}
\usepackage{hyperref}
\usepackage{cleveref}

\usepackage[yyyymmdd]{datetime} % For ISO standard date format
\renewcommand{\dateseparator}{--} % Change / to -- in date format

%% The real stuff
\makeatletter

\makeatother

\begin{document}
\begin{titlepage}
\begin{center}
~\\
~\\
~\\
~\\
\normalsize
~\\
~\\
~\\
~\\
~\\
~\\
~\\
~\\
~\\
\Huge
\textsc{ Tunestarter v0.1 \\ }
~\\
\Large
\textsc{
Cheat sheet music to get the session going\ldots \\
}
~\\
\normalsize
\ldots or at least you'll look organized!
~\\
~\\
~\\
~\\
~\\
~\\
~\\
~\\
~\\
~\\
~\\
\normalsize
\textsc{
Contributors:
}\\
Pär Persson Mattsson\\
~\\
~\\
~\\
~\\
~\\
~\\
~\\
~\\
~\\
~\\
~\\
Last update\\
\today{}

\end{center}
\end{titlepage}

\newgeometry{top=0.5in,bottom=0.5in,right=0.5in,left=0.5in}
\setcounter{tocdepth}{3}
\tableofcontents
%\listoffigures
%\listoftables
\addtocontents{toc}{~\hfill\textbf{Page}\par}
\clearpage
%\pagestyle{headings}

\setcounter{page}{1}
\pagenumbering{arabic}
\chapter*{Some explanation}
Tunes are hard to remember.

And it is even harder to remember them when they are in sets.

And then you have to remember what order they are played in, and of course, how the darn tunes start. The goal of this document is to help you with this.

In the sets section, you only see the first row of each tune, just to get you started. Each tune is also displayed fully in the tunes section.

In the tunes section, you will find the full tunes, with cross references in what set they are listed.

So\ldots

Enjoy!

Copyright notice: I do not claim to have written any of the tunes in this compendium. This is simply a collection, mostly with data from thesession.org. 

\chapter{Sets}



\chapter{Tunes}



\printindex
\end{document}